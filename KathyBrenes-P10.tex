% Instituto Tecnologico de Costa Rica
% Maestría en Ciencias de la Computación
% Introducción a la Investigación
% II Semestre 2018
% 
% Author: Kathy Brenes Guerrero. 
% Profesor: Francisco Torres Rojas. 
% Chapter 3: Review of the Related Literature
%

\documentclass{beamer}
\usepackage{multicol}
\setbeamertemplate{navigation symbols}{}

\usetheme{Warsaw}

\newcounter{sauvegardeenumi}
\newcommand{\asuivre}{\setcounter{sauvegardeenumi}{\theenumi}}
\newcommand{\suite}{\setcounter{enumi}{\thesauvegardeenumi}}

\beamersetuncovermixins{\opaqueness<1>{25}}{\opaqueness<2->{15}}
\begin{document}
\title{Investigaci\'on pr\'actica: Planificaci\'on y dise\~no} 
\subtitle{Revisi\'on de la literatura relacionada}   
\author{Autores: Paul D.Leedy, Jeanne Ellis Ormrod.}
\institute{ 
    Introducci\'on a la Investigaci\'on 
    \newline Instituto Tecnol\'ogico de Costa Rica 
    \newline Estudiante: Kathy Brenes Guerrero.
} 
\date{\today} 


\begin{frame}
\titlepage
\end{frame}

\begin{frame}
\frametitle{Tabla de contenidos}  
    \tableofcontents[hideallsubsections]
\end{frame} 


\section{Revisi\'on de la literatura} %Comprender el papel de la revisi\'on de la literatura
    \begin{frame}
        \frametitle{Comprender el papel de la revisi\'on de la literatura} 
        \begin{multicols}{2}
          
                \begin{enumerate}
                    \item Determinar el estado actual.
                        \begin{figure}
                            \includegraphics[scale=0.50]{figures/prework}
                            \newline
                            {\tiny Tomado de la referencia [1] }
                            \caption{ Trabajo previo. }
                        \end{figure}
            
                    \item Ofrecer nuevas ideas, perspectivas y enfoques.
                        \begin{figure}
                            \includegraphics[scale=0.15]{figures/newIdeas}
                            \newline
                            {\tiny Tomado de la referencia [2] }
                            \caption{Nuevas ideas.}
                        \end{figure}
                        \asuivre
                \end{enumerate}
            \end{multicols}
    \end{frame}


    \begin{frame}
        \frametitle{Comprender el papel de la revisi\'on de la literatura} 
        \begin{multicols}{2}
            \begin{enumerate}
                \suite
                \item Posibles personas para recibir asesoramiento.
                \begin{figure}
                    \includegraphics[scale=0.10]{figures/asesoramiento}
                    \newline
                    {\tiny Tomado de la referencia [3] }
                    \caption{ Asesoramiento. }
                \end{figure}

                \item Metodolog\'ia de trabajo de otros.
                \item Fuentes de datos que no conoc\'ia.
                    \begin{figure}
                        \includegraphics[scale=0.40]{figures/newResources}
                        \newline
                        {\tiny Tomado de la referencia [4] }
                        \caption{Nuevas referencias.}
                    \end{figure}
                    \asuivre
            \end{enumerate}
        \end{multicols}
    \end{frame}

     \begin{frame}
        \frametitle{Comprender el papel de la revisi\'on de la literatura} 
        \begin{multicols}{2}
            \begin{enumerate}
                \suite
                \item Herramientas de medici\'on de otros investigadores.
                \begin{figure}
                    \includegraphics[scale=0.50]{figures/medicion}
                    \newline
                    {\tiny Tomado de la referencia [5] }
                    \caption{Gr\'aficos de resultados. }
                \end{figure}

                \item M\'etodos para lidiar con dificultades.
                \item Interpretar y dar sentido a los resultados.
                    \begin{figure}
                        \includegraphics[scale=0.70]{figures/resultsOriented}
                        \newline
                        {\tiny Tomado de la referencia [6] }
                        \caption{Entender resultados.}
                    \end{figure}
                    \asuivre
            \end{enumerate}
        \end{multicols}
    \end{frame}

     \begin{frame}
        \frametitle{Comprender el papel de la revisi\'on de la literatura} 
            \begin{enumerate}
                \suite
                \item Reforzar la confianza en que vale la pena estudiar su tema.
                \begin{multicols}{2}
                    \begin{figure}
                        \includegraphics[scale=0.50]{figures/confianza}
                        \newline
                        {\tiny Tomado de la referencia [7] }
                        \caption{ Confianza. }
                    \end{figure}

                    \begin{figure}
                        \includegraphics[scale=0.50]{figures/recursos}
                        \newline
                        {\tiny Tomado de la referencia [8] }
                        \caption{ Recursos invertidos. }
                    \end{figure}
                \end{multicols}
            \end{enumerate}       
    \end{frame}

\section{Localizar literatura relacionada} 
    \begin{frame}
    \frametitle{Estrategias para localizar literatura relacionada}
        \begin{multicols}{2}            
            \begin{itemize}
                \item Muchos recursos disponibles.  
                \item Enfocar la b\'usqueda. 
                \item \textbf{Palabras clave.} 
                \item Use libros y publicaciones con fechas de copyright recientes.                    
            \end{itemize}                 
            \begin{figure}
                \includegraphics[scale=0.60]{figures/bullying}
                \newline
                {\tiny Tomado de la referencia [9] }
                \caption{ Ejemplo de Bullying. }
            \end{figure}                    
        \end{multicols}
    \end{frame}

    \subsection{Cat\'alogo de la biblioteca}
        \begin{frame}\frametitle{Cat\'alogo de la biblioteca}
            \begin{itemize}
                \item Publicaciones peri\'odicas que posee la biblioteca.                
                \item Sistema de clasificaci\'on Dewey Decimal (DD).
                \item Sistema de clasificaci\'on de la Biblioteca del Congreso (LC).                 
            \end{itemize} 
            \begin{figure}
                \includegraphics[scale=0.15]{figures/catalogoBiblioteca}
                \newline
                {\tiny Tomado de la referencia [10] }
                \caption{ Cat\'alogo. }
            \end{figure}      
        \end{frame}

    \subsection{\'Indices and Abstracts}
    \begin{frame}
    \frametitle{\'Indices and Abstracts}
        \begin{multicols}{2}  
            \begin{itemize}
                \item \textbf{\'Indice: } Incluye art\'iculos, informes de investigaci\'on y otros documentos. 
            \end{itemize}
            \begin{itemize}
                \item \textbf{Abstract: } Proporciona la fuente del estudio original. 
            \end{itemize}  
        \end{multicols}
        \begin{figure}
            \includegraphics[scale=0.45]{figures/index}       
        \end{figure}  
    \end{frame}

    \subsection{Bases de datos en l\'inea}
        \begin{frame}\frametitle{Bases de datos en l\'inea}
            \begin{itemize}
                \item Permiten b\'usquedas en miles de revistas y otras fuentes.  
                \item Ubicar fuentes de informaci\'on disponibles en la biblioteca.
            \end{itemize} 
            \begin{figure}
                \includegraphics[scale=0.45]{figures/onlineDB}
                \newline
                {\tiny Tomado de la referencia [11] }
                \caption{ Base de datos en l\'inea. }
            \end{figure} 
        \end{frame}

    \subsection{Bibliotecarios de referencia}
    \begin{frame}\frametitle{Bibliotecarios de referencia}
        \begin{multicols}{2}  
            \begin{itemize}
                \item Demostrar c\'omo usar los recursos. 
                \item La mejor manera de dominar la biblioteca es usarla.  
            \end{itemize} 
             \begin{figure}
                \includegraphics[scale=0.45]{figures/bibliotecario}
                \newline
                {\tiny Tomado de la referencia [12] }
                \caption{Bibliotecario. }
            \end{figure}
        \end{multicols}       
    \end{frame}

    \subsection{Navegar por Internet}
    \begin{frame}\frametitle{Navegar por Internet}
    \begin{multicols}{2}  
        \begin{itemize}
            \item Indicaciones importantes. 
                \begin{enumerate}
                    \item Usar al menos dos palabras clave.
                    \item Escriba un signo m\'as (+) antes de cualquier palabra clave.
                    \item Ponga comillas alrededor de la frase.
                \end{enumerate} 
            \item Siempre que encuentre informaci\'on, debe anotar el URL y la fecha.
        \end{itemize} 
        \begin{figure}
                \includegraphics[scale=0.25]{figures/navWeb}
                \newline
                {\tiny Tomado de la referencia [13] }
                \caption{Navegar por internet. }
            \end{figure}
        \end{multicols}
    \end{frame}

   \begin{frame}
   \frametitle{Uso de citas y listas de referencias de los que se han ido antes}
        \begin{itemize}
            \item Valiosa orientaci\'on sobre estudios de investigaci\'on.  
            \item Rastrear cualquier referencia que vea citada por tres o m\'as investigadores.
            \item \textbf{Siempre que sea posible, vaya a la fuente original y l\'eala usted mismo.}
        \end{itemize} 
    \end{frame}

\section{B\'usqueda de literatura} 
 \begin{frame}\frametitle{B\'usqueda de literatura}
    \begin{multicols}{2}  
        \begin{itemize}
            \item Seleccionar un problema o pregunta de investigaci\'on.
            \item Enfoque de divide y vencer\'as.
        \end{itemize} 
        \begin{figure}
                \includegraphics[scale=0.25]{figures/recordar}
                \newline
                {\tiny Tomado de la referencia [14] }
                \caption{Importante de recordar. }
            \end{figure}
        \end{multicols}
    \end{frame}

    \subsection{Utilizar el tiempo de biblioteca de manera eficiente}
        \begin{frame}\frametitle{Usar su tiempo de biblioteca de manera eficiente}
        \begin{multicols}{2}
            \begin{enumerate}
                \item Disponer de herramientas de recopilaci\'on de datos.
                \item Identificar el material y la disponibilidad.
                \item Desarrollar un plan de ataque. 
                \item Busque sus fuentes.            
                \asuivre
            \end{enumerate}
             \begin{figure}
                \includegraphics[scale=0.45]{figures/timeManagement}
                \newline
                {\tiny Tomado de la referencia [15] }
                \caption{Manejo del tiempo. }
            \end{figure}
        \end{multicols}
        \end{frame}
        \begin{frame}\frametitle{Usar su tiempo de biblioteca de manera eficiente}
        \begin{multicols}{2}
            \begin{enumerate}
                \suite
                \item Registre toda la informaci\'on b\'asica mientras lee cada fuente.
                \item Identifique estrategias para obtener fuentes que no est\'an disponibles.            
                \asuivre
            \end{enumerate}
             \begin{figure}
                \includegraphics[scale=0.20]{figures/timeManagement2}
                \newline
                {\tiny Tomado de la referencia [16] }
                \caption{Manejo de los recursos. }
            \end{figure}
        \end{multicols}
         \end{frame}

    \subsection{Evaluar la investigaci\'on de otros}
        \begin{frame}
        \frametitle{Evaluar la investigaci\'on de otros}
        \begin{multicols}{2}
            \begin{enumerate}
                \item Determinar ideas, hallazgos de investigaci\'on y conclusiones.
                \item Conciliar los hallazgos inconsistentes obtenidos.
                \item Ideas sobre c\'omo puede mejorar sus propios esfuerzos.        
                \asuivre
            \end{enumerate}
             \begin{figure}
                \includegraphics[scale=0.45]{figures/checklist}
                \newline
                {\tiny Tomado de la referencia [17] }
                \caption{Evaluar. }
            \end{figure}
        \end{multicols}
        \end{frame}

\section{Escribir la rese\~na de la literatura}
    \begin{frame}\frametitle{Escribir la rese\~na de la literatura}
    \begin{multicols}{2}
                \begin{enumerate}                    
                    \item Obtener la orientaci\'on psicol\'ogica adecuada.
                    \item Tener un plan.  
                    \item Enfatizar la relaci\'on con su problema de investigaci\'on.         
                    \asuivre
                \end{enumerate}
                \begin{figure}
                    \includegraphics[scale=0.10]{figures/focus}
                    \newline
                    {\tiny Tomado de la referencia [18] }
                    \caption{Manejo de los recursos. }
                \end{figure}
            \end{multicols}
    \end{frame}

    \begin{frame}\frametitle{Escribir la rese\~na de la literatura}
    \begin{multicols}{2}
                \begin{enumerate}
                    \suite
                    \item Utilizar frases de transici\'on.
                    \item Diferenciar entre escribir la literatura y plagiarla.  
                    \item Siempre dar cr\'edito.       
                    \item Minimice el uso de citas directas del trabajo de otras personas.    
                    \asuivre
                \end{enumerate}               
            \end{multicols}
    \end{frame}

    \begin{frame}\frametitle{Escribir la rese\~na de la literatura}
    \begin{multicols}{2}
                \begin{enumerate}
                    \suite
                    \item Resuma lo que dijo.
                    \item Recuerde que su primer borrador seguramente no ser\'a su \'ultimo borrador. 
                    \item Pida consejos y comentarios a los dem\'as.                   
                \end{enumerate}               
            \end{multicols}
    \end{frame}

\section{Conclusiones}
    \begin{frame}\frametitle{Conclusiones}
        \begin{enumerate}                    
            \item Seleccione un tema que sea de su inter\'es.
            \item Haga un trabajo de investigaci\'on previa antes de comenzar a trabajar. 
            \item Recuerde que hacer una buena revisi\'on de la literatura puede ahorrarle mucho tiempo tiempo y trabajo.                   
        \end{enumerate}   
    \end{frame}

    \begin{frame}%%     2
    \begin{center}
    {\fontsize{25}{30}\selectfont Muchas gracias.}
    \end{center}
    \end{frame}
\end{document}